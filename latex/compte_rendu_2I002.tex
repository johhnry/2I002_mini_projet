\documentclass[11pt,a4paper]{article}
\usepackage[utf8]{inputenc}
\usepackage[french]{babel}
\usepackage[T1]{fontenc}
\usepackage{amsmath}
\usepackage{amsfonts}
\usepackage{amssymb}
\usepackage{listings}
\usepackage{geometry}
\geometry{left=3.5cm, right=3.5cm}
\lstset{language=Java}

\begin{document}
\title{Compte rendu projet 2I002}
\author{Isabelle Bernard,\\
	Joseph Henry,\\
   Sorbonne Université,\\
   Paris,\\
   France,\\
   L2 informatique\\
   \texttt{joseph.henry@ensci.com}\\
   \texttt{songhui-isabelle@ensci.com}}
\date{\today}
\maketitle

\vspace{5\baselineskip}
\begin{abstract}
Notre projet vise à réaliser une simulation simplifiée de la terre et des ressources dont elle est composée.
\end{abstract}

\newpage

\part{Introduction}
\paragraph{Notre projet est une simulation de l'écosystème terrestre. Cette simulation tente de simplifier et de structurer les éléments et les interactions de cet écosystème.}
\paragraph{Le but de ce projet est de proposer une application graphique et ludique pour inviter les utilsateurs à expérimenter, construire et se rendre compte de l'impact de leurs actions.}

\part{Execution du Main}
\paragraph{La fonction main (\texttt{Simulation.java}) réalise les étapes suivantes : \break}
\begin{enumerate}
\item{Appel de la librairie Processing}
\begin{lstlisting}
PApplet.main("main.Simulation");
\end{lstlisting}
\item{Ouverture de la fenêtre graphique (utilisation de la librairie \texttt{OpenGL})}
\begin{lstlisting}
fullScreen(P3D);
\end{lstlisting}
\item{Initialisation du terrain, de l'interface graphique et la caméra (librairie \texttt{PeasyCam})}
\begin{lstlisting}
terrain = new Terrain(this,  width/2, height/2, 0, 180);
cp5 = new ControlP5(this);
...
\end{lstlisting}
\item{Chargement des modèles}
\begin{lstlisting}
try {
	loadModels(models);
}catch(NumberFormatException e){
	System.out.println("Erreur de chargement du modele");
}
\end{lstlisting}
\newpage
\item{La fonction \texttt{void draw()} s'execute en boucle (environ 30 fois par seconde)}
\item{On affiche et on met à jour les différents composants}
\begin{lstlisting}
soleil.display();
terrain.display();
cp5.draw();
...
options.update();
terrain.update();
soleil.update();
\end{lstlisting}
\end{enumerate}

\part{Difficultés rencontrées}
\begin{itemize}
\item Grande arborescence de fichiers et de classes.
\\$\rightarrow$ Diagramme UML
\item Optimisation et chargement des fichiers 3d.
\\$\rightarrow$ Utilisation d'un dictionnaire
\item Question de l'échelle de la simulation (taille, quantités, plage de valeurs...).
\\$\rightarrow$ Intuition
\item Gestion de l'héritage et transmission des attributs et des méthodes sur plusieurs générations.
\\$\rightarrow UML$
\item Géométrie dans l'espace et rotations tridimensionnelles.
\\$\rightarrow$ Tentative de résolution par des moyens mathématiques (via forum) mais le problème s'est résolu par une réorganisation du code
\end{itemize}

\part{Travaux futurs}
\begin{itemize}
\item Création de connexions entre des villes. (implique un défi géométrique).
\item Ajout d'une ambience sonore 
\item Sécuriser et factoriser d'avantage le code.
\end{itemize}

\end{document}